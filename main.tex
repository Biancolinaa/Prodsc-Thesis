\documentclass[italian,master]{unibg}

\usepackage{color}
\usepackage{amsmath}
\usepackage{titlesec}

\newcommand{\todo}[1]{{\color{red} TODO: #1}}

% ==== FRONTESPIZIO ====
\title{Produzione scientifica}
\subtitle{Analisi e confronto delle principali metriche di valutazione}
\advisor{Chiar.mo Prof.~Stefano Paraboschi}

\department{Ingegneria Gestionale, dell'Informazione e della Produzione}
\course{Ingegneria Informatica}
\class{LM-32}

\author{Bianca Crippa}
\studentid{1053356}
\year{2022/2023}

% ==== FANCY TITLES ====
\definecolor{gray75}{gray}{0.75}
\newcommand{\PRLsep}{\noindent\makebox[\linewidth]{\resizebox{\linewidth}{1.5pt}{\textcolor{gray75}{$\bullet$}}}}
\renewcommand{\headrulewidth}{0pt}
\titleformat{\chapter}[hang]{\centering\Huge\bfseries}{\thechapter.\hspace{10pt}}{0pt}{\Huge\bfseries}[\vspace*{-0.7\baselineskip}\PRLsep]

% ==== CODE SYNTAX HIGHLIGHTING ====

\definecolor{codegreen}{rgb}{0,0.6,0}
\definecolor{codegray}{rgb}{0.5,0.5,0.5}
\definecolor{codeblue}{rgb}{0,0,1}
\definecolor{backcolour}{rgb}{0.95,0.95,0.92}

\lstdefinestyle{mystyle}{
    backgroundcolor=\color{backcolour},   
    commentstyle=\color{codegreen},
    keywordstyle=\color{magenta},
    numberstyle=\tiny\color{codegray},
    stringstyle=\color{codeblue},
    basicstyle=\ttfamily\footnotesize,
    breakatwhitespace=false,         
    breaklines=true,                 
    captionpos=b,                    
    keepspaces=true,                 
    numbers=left,                    
    numbersep=5pt,                  
    showspaces=false,                
    showstringspaces=false,
    showtabs=false,                  
    tabsize=2,
    upquote=true
}

\lstset{style=mystyle}


\begin{document}
\renewcommand{\abstractname}{Abstract}
\maketitle
\emptypage
\begin{abstract}

Questo progetto di tesi consiste in un'analisi di dati riguardanti la valutazione della produzione scientifica.

Sono state prodotte delle metriche per affiliazioni, autori e conferenze, mostrando il forte impatto che conferenze di livello top e grandi affiliazioni hanno su parametri come l'$h$-index di un autore o il numero di citazioni di un paper. 

\end{abstract}
\emptypage
\toc
% \emptypage

\clearpage
\pagenumbering{arabic}

\chapter{Introduzione}

%%%%%% OBIETTIVI %%%%%%%%
\section{Obiettivi del progetto}
Il seguente progetto di tesi si pone l’obiettivo di trovare delle relazioni tra diversi indici bibliometrici che vengono utilizzati 
per valutare la produzione accademica. 
Una visione complessiva dei dati può essere utile per valutare gli andamenti e migliorare la produttività accademica. 
L’applicazione sviluppata permette, tramite interfaccia grafica, di gestire e visualizzare autori, paper, università e conferenze, 
interfacciandosi con un database per l’inserimento e l’interrogazione dei dati. 


%%%%%% CONTESTO %%%%%%%%%

\section{Contesto}
Nell’ambito della pubblicazione di testi di interesse accademico, una pubblicazione scientifica è uno scritto oggettivo riguardante 
un argomento scientifico, redatto da ricercatori o gruppi di ricerca universitari. 

Il gruppo di ricerca rende ufficialmente pubblici i metodi e i risultati delle proprie ricerche sottoponendo il paper ad una 
conferenza oppure pubblicandolo su riviste accademiche. Le pubblicazioni sono regolamentate da procedure di accettazione e valutazione 
per stabilire se il lavoro presentato possieda i requisiti per essere pubblicato.

Il Ministero dell’Istruzione dell’Università e della Ricerca nel documento dell’adunanza del 24/02/2009 che ha come oggetto 
\textit{Criteri identificanti il carattere scientifico delle pubblicazioni, ai sensi dell’art. 3-ter, comma 2, del decreto legge 10 
novembre 2008, n. 180, convertito dalla legge 9 gennaio 2009, n.1.}  considera pubblicazioni scientifiche le seguenti categorie:

\begin{itemize}
    \item Gli articoli pubblicati su riviste scientifiche, che riportano un codice ISSN e che sono stati 
    sottoposti a una procedura di revisione prima della pubblicazione per garantire autorevolezza;
    \item Le monografie di ricerca che riportano codice ISBN e hanno superato la procedura di 
    accettazione per la pubblicazione; 
    \item Gli articoli di ricerca pubblicati in volumi collettivi, sottoposti anch'essi alle stesse procedure di 
    accettazione degli articoli pubblicati su riviste scientifiche;
    \item Tutto ciò che sia riconducibile ad attività di ricerca, come brevetti, software, disegni, 
    purché accompagnato da pubblicazioni e documentazione per poter essere valutati.
\end{itemize}

Inoltre, nel documento dell’adunanza del 22/10/2013 che ha come oggetto \textit{Proposta «Criteri identificanti il carattere 
scientifico 
delle pubblicazioni e degli altri prodotti della ricerca» ai sensi art.3-ter, 
comma 2, l. 9 gennaio 2009, n.1 e successive modificazioni}, vengono proposti quattro criteri che devono essere simultaneamente 
soddisfatti per poter definire una pubblicazione accademica:

\begin{itemize}

    \item I risultati presentati devono essere originali;
    \item I risultati presentati devono poter essere verificati e riutilizzati in altre attività di ricerca;
    \item Il linguaggio deve rendere la pubblicazione accessibile alla maggior parte delle persone interessate;
    \item La sede editoriale deve assicurare l’esistenza di una peer review esterna.

\end{itemize}
Vengono inoltre proposti degli elementi che caratterizzano le categorie di pubblicazioni:

\begin{itemize}
    \item Una pubblicazione è scientifica se:
        \begin{itemize}
            \item Espone in modo sistematico i risultati originali del lavoro di ricerca;
            \item È dotata di riferimenti bibliografici;
            \item Riporta i risultati in modo tale che possano essere verificati e riutilizzati in altre attività di ricerca;
            \item Viene sottoposta a una procedura di revisione;
            \item È presente nelle biblioteche universitarie o è pubblicamente accessibile su piattaforme digitali;
            \item È scritta in una lingua consona per la comunità scientifica di riferimento ed è fruibile 
            ai ricercatori interessati.

        \end{itemize}
    \item Una rivista è scientifica se:
        \begin{itemize}
            \item Aderisce ai criteri generali;
            \item Prevede una procedura di revisione formalizzata;
            \item I revisori che partecipano alla procedura di accettazione sono anonimi;
            \item Garantisce una periodicità regolare delle uscite;
            \item Impone il rispetto degli standard richiesti internazionalmente per la sua indicizzazione.

        \end{itemize}

\end{itemize}

Per quanto riguarda la struttura di una pubblicazione, necessariamente devono apparire i seguenti campi:
\begin{itemize}
    \item \textit{Titolo}: deve fornire un riassunto di ciò che viene trattato nell'articolo;
    \item \textit{Nome degli autori}: solitamente viene elencato seguendo l'ordine alfabetico oppure indicando il nome del ricercatore
    che ha contribuito maggiormente alla ricerca;
    \item \textit{Sommario}: viene fatto un abstract per descrivere gli aspetti fondamentali del lavoro;
    \item \textit{Parole chiave}: lista di parole che si riferiscono agli argomenti trattati nell'articolo;
    \item \textit{Classificazione tematica};
    \item \textit{Introduzione}: breve paragrafo che indica gli scopi della ricerca;
    \item \textit{Metodi}: modi in cui sono stati condotti gli studi. Per rispettare il metodo scientifico è fondamentale che
    le procedure utilizzate siano scientifiche, riproducibili da chiunque e confutabili;
    \item \textit{Risultati}: elenco dei dati scientifici ottenuti;
    \item \textit{Discussione}: interpretazione e analisi oggettiva dei dati;
    \item \textit{Conclusione}: considerazioni ed epilogo del lavoro svolto;
    \item \textit{Riferimenti}: elenco, in ordine alfabetico per nome degli autori, delle note bibliografiche;
    \item \textit{Riconoscimenti, appendici e supplementi}: informazioni accessorie.
\end{itemize}

La differenza tra un documento presentato ad una conferenza rispetto ad uno presentato ad una rivista di ricerca è che il primo è più breve e 
i tempi di revisione sono minori. Il lavoro inviato alle conferenze è, generalmente, limitato alla pubblicazione all’interno della documentazione 
della conferenza ma quando si tratta di ottimi lavori, essi possono essere pubblicati sulle riviste scientifiche.

I ricercatori vengono invitati ad approfondire i temi attuali tramite le call for papers (CFP). Di solito includono il tema e lo scopo della conferenza, 
le linee guida per le presentazioni, i requisiti per le proposte e le scadenze da rispettare. Quando si sottopone un paper ad una conferenza, bisogna 
tener presente che il pubblico a cui ci si riferisce è molto specifico; perciò, la presentazione dell’elaborato deve essere unica, adattata al tema e 
allo scopo della conferenza. 

Prima di sottomettere il paper alla conferenza è necessario scrivere una proposta per l’articolo. È una breve presentazione, simile ad un abstract, che 
deve tenere conto dei requisiti unici per la conferenza. Se la presentazione del paper verrà accettata, sarà possibile presentare il proprio lavoro alla 
conferenza. 

\begin{table}
    \centering
    \begin{tabular}{||c c ||} 
     \hline
     Classe & Descrizione \\ [0.5ex] 
     \hline\hline
     A++, A+ & Conferenze di alto livello \\ 
     \hline
     A, A- & Conferenze di alta qualità \\
     \hline
     B, B- & Conferenze di buona qualità \\ [1ex] 
     \hline
    \end{tabular}
    \caption{Rating delle conferenze}
    \label{table:ratings}
\end{table}

Un indice importante che riguarda le conferenze è il rating, cioè come viene valutata una conferenza. Nella tabella 
\ref{table:ratings} vengono mostrati i rating delle conferenze più importanti. 


%%%% SCOPUS %%%%

\subsection{Scopus}
I dati utilizzati per la costruzione del database di questo progetto di tesi sono stati scaricati da Scopus. 
Esso è una base di dati sviluppata da Elsevier che contiene papers e articoli che riguardano la ricerca in ambito scientifico, 
tecnologico, biomedico e delle scienze sociali. 
Il database viene aggiornato quotidianamente ed è possibile consultare articoli a partire dal 1966. Inoltre permette all'utente, 
collegato ad una rete universitaria, di accedere velocemente agli abstract e ai testi completi.


Alcuni indicatori registrati da scopus per gli autori sono:
\begin{itemize}
  \item \textit{citation-count}: contiene il numero di volte in cui le pubblicazioni di un autore sono state citate in altri articoli 
  di riviste trattate da Scopus;
  \item \textit{cited-by-cout}: indice di citazione a livello di articolo, indica quante citazioni sono state ricevute. 
  Questo indicatore viene utilizzato anche per articoli scientifici e il conteggio viene riportato accanto al riferimento bibliografico;
  \item \textit{h-index}: è un indice che misura la produttività e l'impatto che ha il lavoro pubblicato da un ricercatore. 
  In Scopus questo indice non è statico ma viene calcolato in tempo reale in base alle ricerche fatte. 
  Il calcolo è stato suggerito da Hirsch e può essere riassunto come: 
  \textit{“A scientist has an index h if h of his/her \(N_p\) papers has at least h citations each, and the other \((N_p - h)\) 
  papers have no more than h citations each”}. Questo vuol dire che, per esempio, se un autore ha h-index pari a 4, ha pubblicato 4 lavori 
  citati almeno 4 volte ciascuno.
  Per il calcolo dell’h-index è fondamentale che ci siano i riferimenti agli articoli citati.
\end{itemize}

Alcuni indicatori registrati da scopus per le università sono:
\begin{itemize}
    \item \textit{author-count}: contiene il numero di autori che pubblicano per l'università;
    \item \textit{document-cout}: contiene il numero di paper prodotti dall'università.
  \end{itemize}



%%% MODELLIZZAZIONI %%%%%
\section{Modellizzazioni}
L'analisi del problema ha permesso di individuare quattro entità fondamentali:

\begin{itemize}
    \item \textit{Conference}: rappresenta la conferenza a cui viene sottoposto il paper;
    \item \textit{Affiliation}: rappresenta l'università per cui l'autore ha fatto ricerca. 
    \item \textit{Paper}: rappresenta il paper portato alla conferenza;
    \item \textit{Author}: rappresenta l'autore o gruppo di autori del paper. 
\end{itemize}

Nel modello ER in Figura \ref{fig:er1} sono visibili le relazioni che intercorrono tra le entità.

Un paper può partecipare ad una sola conferenza e può essere scritto da più autori che possono appartenere ad università diverse.
Si viene perciò a creare una tripla tra le entità Paper, Affiliation e Author. 
Per semplificare lo schema, si è introdotta la tabella Pubblication visibile in Figura \ref{fig:er2}, che ha come attributi 
le chiavi primarie delle tre entità precedenti, tenendo comunque in considerazione che un autore, su uno stesso paper, 
può avere affiliation diverse.

Le tabelle presenti nel database, coi relativi attributi e chiavi esterne, sono visibili in Figura \ref{fig:logico}.
La tabella Pubblication permette di effettuare le interrogazioni al database e il collegamento tra le entità Paper, Affiliation e Author.


\begin{figure}[h!]
    \centering
    \includegraphics[width=0.8\linewidth]{./images/er1.png}
    \caption{Modello ER del problema}
    \label{fig:er1}
\end{figure}

\begin{figure}[h!]
    \centering
    \includegraphics[width=0.8\linewidth]{./images/er2.png}
    \caption{Modello ER del problema}
    \label{fig:er2}
\end{figure}


\begin{figure}[h!]
    \centering
    \includegraphics[width=0.8\linewidth]{./images/logico.png}
    \caption{Schema logico del problema}
    \label{fig:logico}
\end{figure}
\chapter{Implementazione}

\section{Descrizione del progetto}

% \todo{Una descrizione del progetto, con la separazione in fasi}
Il progetto si pone l'obiettivo di raccogliere una quantità rappresentativa
di paper, autori e conferenze degli ultimi anni al fine di derivare alcune
correlazioni fra le principali metriche, quali l'H-index, la qualità delle
conferenze ed il numero delle citazioni.

Il lavoro si suddivide principalmente nelle seguenti fasi:
\begin{enumerate}
  \item Modellazione dei dati inerenti al problema;
  \item Raccolta e organizzazione dei dati da varie fonti, quali Scopus e GGS;
  \item Integrazione dei dati in un unico modello al fine di creare un database unificato;
  \item Sviluppo di un'applicazione web in Django per la visualizzazione delle varie metriche.
\end{enumerate}

Oltre agli indici direttamente disponibili (e.g., H-index, citazioni, qualità
delle conferenze), sono state derivate alcune metriche per via campionaria.
Esse sono:
\begin{itemize}
  \item La distribuzione della qualità delle conferenze analizzate;
  \item La correlazione tra qualità delle conferenze e distribuzione del numero delle citazioni dei paper;
  \item \todo{Aggiungere le altre}.
\end{itemize}

%%%% STRUMENTI %%%%%
\section{Strumenti}

I dati che sono stati analizzati in questo progetto di tesi sono stati scaricati
dal database di Scopus utilizzando le API che il sito mette a disposizione.

Per implementare la dashboard e le varie pagine che mostrano i risultati delle
interrogazioni è utilizzato Python come linguaggio di programmazione, il
framework Django e SQLite come database.

%%%% SCOPUS API %%%%%
\subsection{Scopus API}

Scopus permette di accedere a circa 78 milioni di articoli, riviste
scientifiche, libri e pubblicazioni. Le API di Scopus permettono di visualizzare
abstract e dati riguardanti il numero di citazioni di tutte le riviste
accademiche indicizzate da Scopus.

Sono disponibili due modi di utilizzo delle API di Scopus: commerciale e
non commerciale. Per questa tesi sono state utilizzate le API a uso non
commerciale, disponibili gratuitamente per lavori senza scopo di lucro.
Utilizzando le API di Scopus si ha accesso diretto ai dati di Scopus in tempo
reale. Le \textit{API responses} includono link a risorse collegate, rendendo
la navigazione più semplice e la documentazione consente di visualizzare in
anteprima la richiesta e la risposta della API in modo interattivo. Inoltre
l'architettura RESTful permette di avere vantaggi di scalabilità, portabilità
e affidabilità.
Tali API risultano però sottoposte ad alcuni limiti, in quanto sono accessibili
solo da reti di atenei riconosciute da Scopus ed implementano rate limiting\footnote{\url{https://dev.elsevier.com/api_key_settings.html}}.

Una categoria di API importante è rappresentata dalla \textit{Scopus Search}~\cite{scopussearch}.
Essa può essere usata per accedere a dati riguardanti
paper, con relativi autori ed atenei. Ogni risultato fornito è collegato ad un
abstract e tramite link al resto del testo dell'articolo.
Questa API utilizza la sintassi booleana che consente agli utenti di combinare
parole chiave attraverso gli operatori \texttt{AND}, \texttt{OR} e \texttt{NOT},
per poter produrre risultati  più rilevanti. La ricerca booleana viene inviata
tramite il parametro \textit{query} della \textit{query string} e il contenuto
della ricerca inviata deve essere  codificato nell'URL.

\subsection{pybliometrics}\label{sec:pybliometrics}

\texttt{pybliometrics}~\cite{pybliometrics} è una libreria di Python che serve per estrarre e
memorizzare nella cache i dati dal database Scopus.
Essa fornisce un insieme di classi che forniscono un'astrazione dalle chiamate
API di Scopus. Tali classi sono:
\begin{itemize}
	\item \texttt{AbstractRetrieval}: implementa l'API Abstract Retrieval di
	Scopus che consente di estrarre informazioni riguardo gli articoli;
	\item \texttt{AffiliationRetrieval}: implementa l'API Affiliation Retrieval
	di Scopus che fornisce informazioni riguardo gli atenei registrati, come per
	esempio città, paese e membri;
	\item \texttt{AuthorRetrieval}: implementa l'API Author Retrieval di Scopus
	che contiene tutte le informazioni riguardanti un autore;
	\item \texttt{ScopusSearch} implementa l'API di ricerca di Scopus, esegue una
	query per cercare documenti e quindi recupera i record della query.
  \end {itemize}

\subsection{GII-GRIN-SCIE}
\todo{Descrivere}

%%%% DJANGO %%%%%
\subsection{Django}

Django~\cite{django} è un framework web open source scritto in Python. È
un framework ``batteries included'', ovvero tenta di fornire più strumenti
possibile per ridurre sia la quantità di codice da scrivere sia la sua
complessità.

Django si basa sull'architettura \textit{Model, View, Template}, basata sulla più
classica MVC. Essa propone tre diverse entità per la gestione dei dati e dell'interazione
con l'utente, che sono:
\begin{itemize}
	\item \textit{Model}, la descrizione dei dati tramite una classe Python;
	\item \textit{View}, un sistema che processa le richieste degli utenti e
	definisce che dati devono essere presentati in risposta;
	\item \textit{Template}, un template HTML con un proprio linguaggio di templating
	chiamato DTL (Django Template Language) che definisce la rappresentazione
	grafica dei dati.
\end{itemize}
In questa categorizzazione, la \textit{View} di MVT corrisponde al \textit{Controller}
di MVC, mentre il \textit{Template} di MVT corrisponde alla \textit{View} di MVC.

\begin{lstlisting}[language=Python,caption=La classe \texttt{Author},       label=lst:author]
from django.db import models

class Author(models.Model):
    surname = models.CharField(max_length=200)
    name = models.CharField(max_length=200)
    citation_count = models.IntegerField()
    cited_by_count = models.IntegerField()
    h_index = models.IntegerField()
\end{lstlisting}

Django fornisce un ORM (\textit{Object-Relational Mapping}) tramite la
rappresentazione di tabelle con delle classi che derivano dalla classe
predefinita \texttt{Model}, come mostrato nel Listing~\ref{lst:author}. In
tal modo, è possibile interagire con degli oggetti Python, del tutto simili
alle classi, astraendo completamente dalla tecnologia di database utilizzata.
Infatti, con Django è possibile utilizzare diversi DBMS~\cite{djangoDBMS}, tra cui PostgreSQL,
MariaDB, MySQL e SQLite. Quest'ultimo è stato scelto per essere usato in questo
progetto di tesi, in quanto la semplicità e la facilità di setup sono risultate
fondamentali per velocizzare la realizzazione di prototipi.

Le \textit{View} di Django possono essere rappresentate da varie entità nel codice,
ma i modi principali sono le \textit{view generiche} tramite classi fornite dal framework
stesso e delle semplici \textit{funzioni} che lavorano su un parametro rappresentante
la richiesta effettuata dall'utente. Come esempio, nel Listing~\ref{lst:view-author-info}
è rappresentata una view che ottiene le informazioni di un autore specifico,
come salvate su database, aggiungendo inoltre una classificazione in base al
rating delle conferenze a cui ha partecipato.

\begin{lstlisting}[language=Python,caption=La view \texttt{author\_info},label=lst:view-author-info]
def author_info(request, author_id):
    author = get_object_or_404(Author, pk=author_id)
    num_confs = {}

    for pub in author.publication_set.all():
        rating = pub.paper.conference.ggs_rating
        if rating in num_confs:
            num_confs[rating] += 1
        else:
            num_confs[rating] = 1

    return render(request,
                  "author_info.html",
                  {"author": author, "num_confs": num_confs})
\end{lstlisting}

Infine, un \textit{Template} di Django è semplicemente un file HTML che
viene interpretato come codice DTL al momento della costruzione della risposta
inviata dal server al client che ha effettuato la richiesta. Un esempio di tale
template è visibile nel Listing~\ref{lst:django-template}.

\begin{lstlisting}[language=HTML, caption=Template per l'elenco delle conferenze,label=lst:django-template]



Conferences - Produzione Scientifica



<h1>List of Conferences</h1>

<img src="" />

<table>
  <thead>
    <tr>
      <th>Name</th>
      <th>Acronym</th>
      <th>Rating</th>
    </tr>
  </thead>
  <tbody>
    
    <tr>
      <td>
        <a href="">{{ conference.name }}</a>
      </td>
      <td>{{ conference.acronym }}</td>
      <td>{{ conference.ggs_rating }}</td>
    </tr>
    
  </tbody>
</table>

\end{lstlisting}

\todo{Descrivere l'organizzazione delle rotte in Django ed il flusso delle richieste.}

\section{Ottenimento dei dati}

Come già anticipato in Sezione~\ref{sec:scopus}, la maggior parte dei dati è
stata presa da Scopus. In particolare, Scopus ha fornito i dati di circa 80000
papers negli ultimi 4 anni, con i relativi autori ed affiliazioni.
Per le conferenze, non fornendo Scopus informazioni se non nome ed acronimo
delle conferenze, si è fatto uso dei dati indicizzati da GII-GRIN-SCIE~\cite{giigrinscie} (GGS).

Per il download dei dati da Scopus, si è usato \texttt{pybliometrics}
(Sezione~\ref{sec:pybliometrics}) per lo sviluppo di script custom orientati
al download di grandi quantità di dati in modo automatico facendo uso delle
API fornite da Scopus.

Per quanto riguarda le conferenze, invece, i nomi delle conferenze forniti da
GGS non concordano in pieno con quelli presenti su Scopus. A questo fine,
si è manualmente proceduto all'analisi delle conferenze ottenute ed al loro
inserimento all'interno del database del progetto.
Al fine di semplificare e velocizzare il processo, si è fatto uso della
\textit{distanza di Levenshtein} per confrontare i nomi dati dalle due fonti.
La priorità è stata comunque data agli acronimi rispetto ai nomi interi.

\subsection{Distanza di Levenshtein}

L'algoritmo di Levenshtein, introdotta per la prima volta nel
1965~\cite{levenshtein1966} calcola la somiglianza tra due stringhe diverse.
Date due stringhe $a$ e $b$, l'algoritmo misura il numero di modifiche
elementari necessarie per trasformare la stringa $a$ nella stringa $b$.
Tra le operazioni elementari si hanno eliminazione di un carattere,
sostituzione di un carattere con un altro, inserimento di un nuovo carattere.
Più precisamente, definite $a$ e $b$ due stringhe, $|a|$ la lunghezza di una
stringa, la distanza di Levenshtein è definita come
\begin{equation*}
  \text{lev}(a, b) =
  \begin{cases}
    |a| & \text{se } |b| = 0 \\
    |b| & \text{se } |a| = 0 \\
    \text{lev}(\text{tail}(a), b) & \text{se } a_0 = b_0 \\
    1 + \min\begin{cases}
      \text{lev}(\text{tail}(a), b) \\
      \text{lev}(a, \text{tail}(b)) \\
      \text{lev}(\text{tail}(a), \text{tail}(b)) \\
    \end{cases}
    & \text{altrimenti}
  \end{cases}
\end{equation*}
%
Dove $\text{tail}$ rappresenta la coda di una lista, i.e., se $a_i$ è il
carattere in posizione $i$, allora data la stringa $a = a_0a_1\cdots a_n$ si ha
che $\text{tail}(a) = a_1\cdots a_n$.
Nell'implementazione è stata usata la libreria
\texttt{python-Levenshtein}~\cite{pythonLevenshtein}.

%%% DOWNOLAD DEI DATI %%%%
\section{Descrizione dei dati}\label{sec:descrizionedati}

In Figura \ref{fig:logico} è visibile lo schema logico, rappresentante
le tabelle presenti direttamente nel database, coi relativi attributi e chiavi
esterne. Notare la presenza della tabella \textit{Publication}, in sostituzione
alla relazione tripla nominata in Sezione~\ref{sec:modellizzazioni}.

Al fine di identificare univocamente i dati, si è fatto uso dei campi
identificativi forniti direttamente dal database di Scopus. Questo per evitare
ambiguità con un'eventuale generazione automatica a partire dai dati,
soprattutto in caso di omonimie.

\begin{figure}
  \centering
  \includegraphics[width=0.8\linewidth]{logico.png}
  \caption{Schema logico del problema}
  \label{fig:logico}
\end{figure}

\subsection{Affiliation}

Un'affiliazione rappresenta un'università, e per ogni singola entry si ha
interesse a mantenere i seguenti campi:

\begin{itemize}
  \item \texttt{aff\_id}, l'identificativo dell'università;
  \item \texttt{affiliation\_name}, nome dell'università;
  \item \texttt{state} e \texttt{country}, contenenti informazioni riguardanti la posizione geografica dell'università;
  \item \texttt{author\_count}, numero di autori associati all'università;
  \item \texttt{document\_count}, numero di documenti per l'università.
\end{itemize}

\subsection{Author}
Un autore è rappresentato dai seguenti campi:
\begin{itemize}
  \item \texttt{author\_id};
  \item dati per il nominativo, ovvero \texttt{surname} e \texttt{name};
  \item \texttt{citation\_count}, numero totale di elementi citati;
  \item \texttt{cited\_by\_count}, numero totale di autori citanti;
  \item \texttt{h\_index}, h-index dell'autore.
\end{itemize}

\subsection{Conference}

Di una conferenza ci interessa mantenere i seguenti campi:
\begin{itemize}
  \item \texttt{title}, nome della conferenza;
  \item \texttt{acronym};
  \item \texttt{ggs\_rating}, il rate come definito da GGS;
  \item \texttt{num\_papers}, numero di paper accettati dalla conferenza;
  \item \texttt{citation\_count}, numero di citazioni totali di tutti i paper della conferenza.
\end{itemize}

\subsection{Paper}

Un paper viene rappresentato dai seguenti dati:

\begin{itemize}
  \item \texttt{id};
  \item \texttt{authors}, lista di autori del paper (rappresentati dai relativi identificativi);
  \item \texttt{title};
  \item \texttt{year};
  \item \texttt{conference}, la conferenza a cui è stato presentato;
  \item \texttt{cited\_by\_count}, il numero di citazioni;
  \item \texttt{doi}, il suo identificativo DOI.
\end{itemize}

% \section{Implementazione di views e metriche}

\section{Applicazione web}

\subsection{Struttura dei file}

\todo{Descrivere la separazione data da Django nei file e nel codice}

\subsection{Calcolo delle metriche}

\todo{Descrivere come le immagini delle metriche sono ottenute}

\section{Metriche e correlazioni}\label{sec:metriche-correlazioni}

\subsection{Distribuzione delle conferenze}

\subsection{Correlazione tra qualità delle conferenze e distribuzione delle citazioni}
\chapter{Risultati}
\todo{Considerare l'unione con \ref{sec:metriche-correlazioni}}
\chapter{Conclusioni}

Questo progetto di tesi nasce con l'obiettivo di trovare una relazione tra diversi indici bibliometrici che vengono utilizzati per la valutazione della produzione accademica. Le metriche che sono state prodotte confrontano paper, autori, affiliazioni e conferenze, utilizzando dati scaricati da Scopus tramite API. Esse sono poi visualizzate in una dashboard.

Per le conferenze, le metriche analizzate riguardano l'impatto del rating della conferenza su vari parametri come il numero di citazioni, la distribuzione e l'$h$-index. Si è concluso che conferenze con qualità maggiore hanno una minore numerosità di paper sottomessi e solitamente attirano ricercatori con un $h$-index più alto. Per quanto riguarda invece il numero di citazioni non vi è un aumento per conferenze di livello top. 

Per le affiliazioni, le metriche prodotte sono state raggruppate per stato di appartenenza e per aree tematiche, evidenziando che i paesi con il maggior numero di ricercatori sono Cina e Stati Uniti.

Infine per quanto riguarda gli autori, l'analisi ha riguardato prevalentemente il rapporto con l'affiliazione in cui pubblicano i loro paper, mostrando che affiliazioni più grandi hanno maggior peso per quanto riguarda il numero di citazioni di un paper e l'$h$-index dell'autore.

\nocite{*}
\printbibliography[heading=bibintoc]

\chapter*{Ringraziamenti}
In conclusione di questo elaborato, voglio ringraziare il mio relatore Stefano Paraboschi e il Seclab per avermi seguito nello sviluppo del progetto e aver fornito il template per questa tesi.

Ringrazio la mia famiglia e i miei amici per aver condiviso con me questi anni di università.

\end{document}
