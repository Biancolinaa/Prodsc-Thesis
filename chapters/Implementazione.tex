\chapter{Implementazione}

\section{Descrizione del progetto}

% \todo{Una descrizione del progetto, con la separazione in fasi}
Il progetto si pone l'obiettivo di raccogliere una quantità rappresentativa
di paper, autori e conferenze degli ultimi anni al fine di derivare alcune
correlazioni fra le principali metriche, quali l'H-index, la qualità delle
conferenze ed il numero delle citazioni.

Il lavoro si suddivide principalmente nelle seguenti fasi:
\begin{enumerate}
  \item Modellazione dei dati inerenti al problema;
  \item Raccolta e organizzazione dei dati da varie fonti, quali Scopus e GGS;
  \item Integrazione dei dati in un unico modello al fine di creare un database unificato;
  \item Sviluppo di un'applicazione web in Django per la visualizzazione delle varie metriche.
\end{enumerate}

Oltre agli indici direttamente disponibili (e.g., H-index, citazioni, qualità
delle conferenze), sono state derivate alcune metriche per via campionaria.
Esse sono:
\begin{itemize}
  \item La distribuzione della qualità delle conferenze analizzate;
  \item La correlazione tra qualità delle conferenze e distribuzione del numero delle citazioni dei paper;
  \item \todo{Aggiungere le altre}.
\end{itemize}

%%%% STRUMENTI %%%%%
\section{Strumenti}

I dati che sono stati analizzati in questo progetto di tesi sono stati scaricati
dal database di Scopus utilizzando le API che il sito mette a disposizione.

Per implementare la dashboard e le varie pagine che mostrano i risultati delle
interrogazioni è utilizzato Python come linguaggio di programmazione, il
framework Django e SQLite come database.

%%%% SCOPUS API %%%%%
\subsection{Scopus API}

Scopus permette di accedere a circa 78 milioni di articoli, riviste
scientifiche, libri e pubblicazioni. Le API di Scopus permettono di visualizzare
abstract e dati riguardanti il numero di citazioni di tutte le riviste
accademiche indicizzate da Scopus.

Sono disponibili due modi di utilizzo delle API di Scopus: commerciale e
non commerciale. Per questa tesi sono state utilizzate le API a uso non
commerciale, disponibili gratuitamente per lavori senza scopo di lucro.
Utilizzando le API di Scopus si ha accesso diretto ai dati di Scopus in tempo
reale. Le \textit{API responses} includono link a risorse collegate, rendendo
la navigazione più semplice e la documentazione consente di visualizzare in
anteprima la richiesta e la risposta della API in modo interattivo. Inoltre
l'architettura RESTful permette di avere vantaggi di scalabilità, portabilità
e affidabilità.
Tali API risultano però sottoposte ad alcuni limiti, in quanto sono accessibili
solo da reti di atenei riconosciute da Scopus ed implementano rate limiting\footnote{\url{https://dev.elsevier.com/api_key_settings.html}}.

Una categoria di API importante è rappresentata dalla \textit{Scopus Search}~\cite{scopussearch}.
Essa può essere usata per accedere a dati riguardanti
paper, con relativi autori ed atenei. Ogni risultato fornito è collegato ad un
abstract e tramite link al resto del testo dell'articolo.
Questa API utilizza la sintassi booleana che consente agli utenti di combinare
parole chiave attraverso gli operatori \texttt{AND}, \texttt{OR} e \texttt{NOT},
per poter produrre risultati  più rilevanti. La ricerca booleana viene inviata
tramite il parametro \textit{query} della \textit{query string} e il contenuto
della ricerca inviata deve essere  codificato nell'URL.

\subsection{pybliometrics}\label{sec:pybliometrics}

\texttt{pybliometrics}~\cite{pybliometrics} è una libreria di Python che serve per estrarre e
memorizzare nella cache i dati dal database Scopus.
Essa fornisce un insieme di classi che forniscono un'astrazione dalle chiamate
API di Scopus. Tali classi sono:
\begin{itemize}
	\item \texttt{AbstractRetrieval}: implementa l'API Abstract Retrieval di
	Scopus che consente di estrarre informazioni riguardo gli articoli;
	\item \texttt{AffiliationRetrieval}: implementa l'API Affiliation Retrieval
	di Scopus che fornisce informazioni riguardo gli atenei registrati, come per
	esempio città, paese e membri;
	\item \texttt{AuthorRetrieval}: implementa l'API Author Retrieval di Scopus
	che contiene tutte le informazioni riguardanti un autore;
	\item \texttt{ScopusSearch} implementa l'API di ricerca di Scopus, esegue una
	query per cercare documenti e quindi recupera i record della query.
  \end {itemize}

\subsection{GII-GRIN-SCIE}
\todo{Descrivere}

%%%% DJANGO %%%%%
\subsection{Django}

Django~\cite{django} è un framework web open source scritto in Python. È
un framework ``batteries included'', ovvero tenta di fornire più strumenti
possibile per ridurre sia la quantità di codice da scrivere sia la sua
complessità.

Django si basa sull'architettura \textit{Model, View, Template}, basata sulla più
classica MVC. Essa propone tre diverse entità per la gestione dei dati e dell'interazione
con l'utente, che sono:
\begin{itemize}
	\item \textit{Model}, la descrizione dei dati tramite una classe Python;
	\item \textit{View}, un sistema che processa le richieste degli utenti e
	definisce che dati devono essere presentati in risposta;
	\item \textit{Template}, un template HTML con un proprio linguaggio di templating
	chiamato DTL (Django Template Language) che definisce la rappresentazione
	grafica dei dati.
\end{itemize}
In questa categorizzazione, la \textit{View} di MVT corrisponde al \textit{Controller}
di MVC, mentre il \textit{Template} di MVT corrisponde alla \textit{View} di MVC.

\begin{lstlisting}[language=Python,caption=La classe \texttt{Author},       label=lst:author]
from django.db import models

class Author(models.Model):
    surname = models.CharField(max_length=200)
    name = models.CharField(max_length=200)
    citation_count = models.IntegerField()
    cited_by_count = models.IntegerField()
    h_index = models.IntegerField()
\end{lstlisting}

Django fornisce un ORM (\textit{Object-Relational Mapping}) tramite la
rappresentazione di tabelle con delle classi che derivano dalla classe
predefinita \texttt{Model}, come mostrato nel Listing~\ref{lst:author}. In
tal modo, è possibile interagire con degli oggetti Python, del tutto simili
alle classi, astraendo completamente dalla tecnologia di database utilizzata.
Infatti, con Django è possibile utilizzare diversi DBMS~\cite{djangoDBMS}, tra cui PostgreSQL,
MariaDB, MySQL e SQLite. Quest'ultimo è stato scelto per essere usato in questo
progetto di tesi, in quanto la semplicità e la facilità di setup sono risultate
fondamentali per velocizzare la realizzazione di prototipi.

Le \textit{View} di Django possono essere rappresentate da varie entità nel codice,
ma i modi principali sono le \textit{view generiche} tramite classi fornite dal framework
stesso e delle semplici \textit{funzioni} che lavorano su un parametro rappresentante
la richiesta effettuata dall'utente. Come esempio, nel Listing~\ref{lst:view-author-info}
è rappresentata una view che ottiene le informazioni di un autore specifico,
come salvate su database, aggiungendo inoltre una classificazione in base al
rating delle conferenze a cui ha partecipato.

\begin{lstlisting}[language=Python,caption=La view \texttt{author\_info},label=lst:view-author-info]
def author_info(request, author_id):
    author = get_object_or_404(Author, pk=author_id)
    num_confs = {}

    for pub in author.publication_set.all():
        rating = pub.paper.conference.ggs_rating
        if rating in num_confs:
            num_confs[rating] += 1
        else:
            num_confs[rating] = 1

    return render(request,
                  "author_info.html",
                  {"author": author, "num_confs": num_confs})
\end{lstlisting}

Infine, un \textit{Template} di Django è semplicemente un file HTML che
viene interpretato come codice DTL al momento della costruzione della risposta
inviata dal server al client che ha effettuato la richiesta. Un esempio di tale
template è visibile nel Listing~\ref{lst:django-template}.

\begin{lstlisting}[language=HTML, caption=Template per l'elenco delle conferenze,label=lst:django-template]



Conferences - Produzione Scientifica



<h1>List of Conferences</h1>

<img src="" />

<table>
  <thead>
    <tr>
      <th>Name</th>
      <th>Acronym</th>
      <th>Rating</th>
    </tr>
  </thead>
  <tbody>
    
    <tr>
      <td>
        <a href="">{{ conference.name }}</a>
      </td>
      <td>{{ conference.acronym }}</td>
      <td>{{ conference.ggs_rating }}</td>
    </tr>
    
  </tbody>
</table>

\end{lstlisting}

\todo{Descrivere l'organizzazione delle rotte in Django ed il flusso delle richieste.}

\section{Ottenimento dei dati}

Come già anticipato in Sezione~\ref{sec:scopus}, la maggior parte dei dati è
stata presa da Scopus. In particolare, Scopus ha fornito i dati di circa 80000
papers negli ultimi 4 anni, con i relativi autori ed affiliazioni.
Per le conferenze, non fornendo Scopus informazioni se non nome ed acronimo
delle conferenze, si è fatto uso dei dati indicizzati da GII-GRIN-SCIE~\cite{giigrinscie} (GGS).

Per il download dei dati da Scopus, si è usato \texttt{pybliometrics}
(Sezione~\ref{sec:pybliometrics}) per lo sviluppo di script custom orientati
al download di grandi quantità di dati in modo automatico facendo uso delle
API fornite da Scopus.

Per quanto riguarda le conferenze, invece, i nomi delle conferenze forniti da
GGS non concordano in pieno con quelli presenti su Scopus. A questo fine,
si è manualmente proceduto all'analisi delle conferenze ottenute ed al loro
inserimento all'interno del database del progetto.
Al fine di semplificare e velocizzare il processo, si è fatto uso della
\textit{distanza di Levenshtein} per confrontare i nomi dati dalle due fonti.
La priorità è stata comunque data agli acronimi rispetto ai nomi interi.

\subsection{Distanza di Levenshtein}

L'algoritmo di Levenshtein, introdotta per la prima volta nel
1965~\cite{levenshtein1966} calcola la somiglianza tra due stringhe diverse.
Date due stringhe $a$ e $b$, l'algoritmo misura il numero di modifiche
elementari necessarie per trasformare la stringa $a$ nella stringa $b$.
Tra le operazioni elementari si hanno eliminazione di un carattere,
sostituzione di un carattere con un altro, inserimento di un nuovo carattere.
Più precisamente, definite $a$ e $b$ due stringhe, $|a|$ la lunghezza di una
stringa, la distanza di Levenshtein è definita come
\begin{equation*}
  \text{lev}(a, b) =
  \begin{cases}
    |a| & \text{se } |b| = 0 \\
    |b| & \text{se } |a| = 0 \\
    \text{lev}(\text{tail}(a), b) & \text{se } a_0 = b_0 \\
    1 + \min\begin{cases}
      \text{lev}(\text{tail}(a), b) \\
      \text{lev}(a, \text{tail}(b)) \\
      \text{lev}(\text{tail}(a), \text{tail}(b)) \\
    \end{cases}
    & \text{altrimenti}
  \end{cases}
\end{equation*}
%
Dove $\text{tail}$ rappresenta la coda di una lista, i.e., se $a_i$ è il
carattere in posizione $i$, allora data la stringa $a = a_0a_1\cdots a_n$ si ha
che $\text{tail}(a) = a_1\cdots a_n$.
Nell'implementazione è stata usata la libreria
\texttt{python-Levenshtein}~\cite{pythonLevenshtein}.

%%% DOWNOLAD DEI DATI %%%%
\section{Descrizione dei dati}\label{sec:descrizionedati}

In Figura \ref{fig:logico} è visibile lo schema logico, rappresentante
le tabelle presenti direttamente nel database, coi relativi attributi e chiavi
esterne. Notare la presenza della tabella \textit{Publication}, in sostituzione
alla relazione tripla nominata in Sezione~\ref{sec:modellizzazioni}.

Al fine di identificare univocamente i dati, si è fatto uso dei campi
identificativi forniti direttamente dal database di Scopus. Questo per evitare
ambiguità con un'eventuale generazione automatica a partire dai dati,
soprattutto in caso di omonimie.

\begin{figure}
  \centering
  \includegraphics[width=0.8\linewidth]{logico.png}
  \caption{Schema logico del problema}
  \label{fig:logico}
\end{figure}

\subsection{Affiliation}

Un'affiliazione rappresenta un'università, e per ogni singola entry si ha
interesse a mantenere i seguenti campi:

\begin{itemize}
  \item \texttt{aff\_id}, l'identificativo dell'università;
  \item \texttt{affiliation\_name}, nome dell'università;
  \item \texttt{state} e \texttt{country}, contenenti informazioni riguardanti la posizione geografica dell'università;
  \item \texttt{author\_count}, numero di autori associati all'università;
  \item \texttt{document\_count}, numero di documenti per l'università.
\end{itemize}

\subsection{Author}
Un autore è rappresentato dai seguenti campi:
\begin{itemize}
  \item \texttt{author\_id};
  \item dati per il nominativo, ovvero \texttt{surname} e \texttt{name};
  \item \texttt{citation\_count}, numero totale di elementi citati;
  \item \texttt{cited\_by\_count}, numero totale di autori citanti;
  \item \texttt{h\_index}, h-index dell'autore.
\end{itemize}

\subsection{Conference}

Di una conferenza ci interessa mantenere i seguenti campi:
\begin{itemize}
  \item \texttt{title}, nome della conferenza;
  \item \texttt{acronym};
  \item \texttt{ggs\_rating}, il rate come definito da GGS;
  \item \texttt{num\_papers}, numero di paper accettati dalla conferenza;
  \item \texttt{citation\_count}, numero di citazioni totali di tutti i paper della conferenza.
\end{itemize}

\subsection{Paper}

Un paper viene rappresentato dai seguenti dati:

\begin{itemize}
  \item \texttt{id};
  \item \texttt{authors}, lista di autori del paper (rappresentati dai relativi identificativi);
  \item \texttt{title};
  \item \texttt{year};
  \item \texttt{conference}, la conferenza a cui è stato presentato;
  \item \texttt{cited\_by\_count}, il numero di citazioni;
  \item \texttt{doi}, il suo identificativo DOI.
\end{itemize}

% \section{Implementazione di views e metriche}

\section{Applicazione web}

\subsection{Struttura dei file}

\todo{Descrivere la separazione data da Django nei file e nel codice}

\subsection{Calcolo delle metriche}

\todo{Descrivere come le immagini delle metriche sono ottenute}

\section{Metriche e correlazioni}\label{sec:metriche-correlazioni}

\subsection{Distribuzione delle conferenze}

\subsection{Correlazione tra qualità delle conferenze e distribuzione delle citazioni}