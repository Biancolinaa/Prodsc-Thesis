\chapter{Conclusioni}

Questo progetto di tesi nasce con l'obiettivo di trovare una relazione tra diversi indici bibliometrici che vengono utilizzati per la valutazione della produzione accademica. Le metriche che sono state prodotte confrontano paper, autori, affiliazioni e conferenze, utilizzando dati scaricati da Scopus tramite API. Esse sono poi visualizzate in una dashboard.

Per le conferenze, le metriche analizzate riguardano l'impatto del rating della conferenza su vari parametri come il numero di citazioni, la distribuzione e l'$h$-index. Si è concluso che conferenze con qualità maggiore hanno una minore numerosità di paper sottomessi e solitamente attirano ricercatori con un $h$-index più alto. Per quanto riguarda invece il numero di citazioni non vi è un aumento per conferenze di livello top. 

Per le affiliazioni, le metriche prodotte sono state raggruppate per stato di appartenenza e per aree tematiche, evidenziando che i paesi con il maggior numero di ricercatori sono Cina e Stati Uniti.

Infine per quanto riguarda gli autori, l'analisi ha riguardato prevalentemente il rapporto con l'affiliazione in cui pubblicano i loro paper, mostrando che affiliazioni più grandi hanno maggior peso per quanto riguarda il numero di citazioni di un paper e l'$h$-index dell'autore.